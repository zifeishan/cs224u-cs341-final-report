Authorship attribution, the task of identifying the author of a document, has been applied to works of historical value such as Shakespeare's plays or the political Federalist papers but is still highly significant in identifying users across online communities, detecting impersonation attacks, or ensuring anonymity in double-blind conference submission processes.
We introduce the novel problem of authorship attribution in multi-authored documents and focus on scientific publications.
We demonstrate using a stylometric approach that paper authors can be predicted with significant accuracy by exploiting authors' stylistic idiosyncrasies.
This challenges the assumption that simply removing names from a paper submission ensures anonymity in a double-blind process.
Multi-authored documents present hard challenges for authorship attribution.
We propose several ideas how these can be addressed and evaluate when such models perform well.
To this end we present a sentence-based prediction model that also allows to estimate which sentences were contributed by which author.