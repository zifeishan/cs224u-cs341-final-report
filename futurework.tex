% !TEX root = cs224_final_paper.tex

In the future, we propose to conduct more in-depth analysis of
individual features, iterate on feature engineering to get more
accurate predictions, and understand he impact of different knowledge
in authorship attribution.

To address on the gap between sentence-level and paper-level
predictions in real-world datasets, we will conduct error analysis on
a real dataset with ground truth annotation on sentence level, to
investigate where our method fails.

We will further look into a more representative model such as
Conditional Random Fields (CRFs), to integrate more knowledge such as
citations, co-authorship correlation, and domain knowledge in
scientific writing. For example, how does citation network help
identifying authors? How do statistics of co-authorships help in
jointly predicting multiple authors in a publication? How to integrate
specific knowledge such as ordering of authors, positions of
sentences, flow and structure in papers, and multiple revisions of
sentences?

Further, we want to study the robustness of our models and the impact
of author pool size to the task: how many papers for each author do we
need to train a good model for an author? How does increasing the
number of authors affect the difficulty for prediction?
